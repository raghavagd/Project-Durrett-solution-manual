\documentclass[a4paper,english,10pt]{article}
\usepackage{%
	amsfonts,%
	amsmath,%	
	amssymb,%
	amsthm,%
	algorithm,%
%	babel,%
	bbm,%
	etex,%
	%biblatex,%
	caption,%
	centernot,%
	color,%
	dsfont,%
	enumerate,%
	epsfig,%
	epstopdf,%
	geometry,%
	graphicx,%
	hyperref,%
	latexsym,%
	mathtools,%
	multicol,%
	pgf,%
	pgfplots,%
	pgfplotstable,%
	pgfpages,%
	proof,%
	psfrag,%
	subfigure,%	
	tikz,%
	ulem,%
	url%
}	
\usepackage[noend]{algpseudocode}
\usepackage[mathscr]{eucal}
\usepgflibrary{shapes}
\usetikzlibrary{%
  	arrows,%
	backgrounds,%
	chains,%
	decorations.pathmorphing,% /pgf/decoration/random steps | erste Graphik
	decorations.text,%
	matrix,%
  	positioning,% wg. " of "
  	fit,%
	patterns,%
  	petri,%
	plotmarks,%
  	scopes,%
	shadows,%
  	shapes.misc,% wg. rounded rectangle
  	shapes.arrows,%
	shapes.callouts,%
  	shapes%
}

\theoremstyle{plain}
\newtheorem{thm}{Theorem}[section]
\newtheorem{lem}[thm]{Lemma}
\newtheorem{prop}[thm]{Proposition}
\newtheorem{cor}[thm]{Corollary}


\theoremstyle{definition}
\newtheorem{defn}[thm]{Definition}
\newtheorem{conj}[thm]{Conjecture}
\newtheorem{exmp}[thm]{Example}
\newtheorem{assum}[thm]{Assumption}
\newtheorem{axiom}[thm]{Axiom}

\theoremstyle{remark}
\newtheorem{rem}{Remark}
\newtheorem{note}{Note}
\newtheorem{fact}{Fact}
\newtheorem{claim}{Claim}

\newcommand{\norm}[1]{\left\lVert#1\right\rVert}
\newcommand{\indep}{\!\perp\!\!\!\perp}
\DeclarePairedDelimiter\abs{\lvert}{\rvert}%
\newcommand\numberthis{\addtocounter{equation}{1}\tag{\theequation}}
\newcommand{\tr}{\operatorname{tr}}
\newcommand{\R}{\mathbb{R}}
\newcommand{\N}{\mathbb{N}}
\newcommand{\E}{\mathbb{E}}
\newcommand{\Z}{\mathbb{Z}}
\newcommand{\B}{\mathscr{B}}
\newcommand{\C}{\mathcal{C}}
\newcommand{\T}{\mathscr{T}}
\newcommand{\F}{\mathcal{F}}
\newcommand{\G}{\mathcal{G}}
%\newcommand{\ba}{\begin{align*}}
%\newcommand{\ea}{\end{align*}}
\newcommand{\expect}[1]{\mathbb{E}\left[{#1}\right]}
\newcommand{\prob}[1]{\mathbb{P}\left[{#1}\right]}
\newcommand{\probo}[1]{\mathbb{P}_0\left[{#1}\right]}
\newcommand{\probi}[1]{\mathbb{P}_1\left[{#1}\right]}
\newcommand{\given}{\; \big\vert \;} 
\newcommand{\bydef}{:=}
\newcommand{\indic}[1]{\mathbbm{1}\{#1\}}
\DeclareMathOperator*{\argmax}{arg\,max}
\renewcommand{\qedsymbol}{$\blacksquare$}
\makeatletter
\def\BState{\State\hskip-\ALG@thistlm}
\makeatother

\makeatletter
\def\th@plain{%
  \thm@notefont{}% same as heading font
  \itshape % body font
}
\def\th@definition{%
  \thm@notefont{}% same as heading font
  \normalfont % body font
}
\makeatother
\date{}
\usepackage{etex,enumitem,hyperref,tikz,pgfplots}
%opening

\title{Solutions to Durrett's Probability: Theory and Examples \\(Edition 4.1)}
%\author{Deadline}
\date{}
\begin{document}
\maketitle
\section*{Chapter 1}
\begin{enumerate}
\item[1.1.1]
\begin{proof} 
We need to show if $\mathcal{F}_i, i\in I$ are $\sigma$ algebras, $\cap_{i \in I} \mathcal{F}_i$ is also a $sigma$ algebra
\begin{enumerate}
\item Since $\Omega \in \mathcal{F}_i$, $ \forall i\in I$  from the definition of $\sigma$ algebra, we have
$\Omega \in \cap_{i \in I} \mathcal{F}_i$
\item If $A\in \cap_{i \in I} \mathcal{F}_i$, then $A$ must be in each of the $\mathcal{F}_is$ i.e. 
$A\in \mathcal{F}_i$, $ \forall i$. Since $\mathcal{F}_is$ are $\sigma$ algebras, we have $A^c \in \mathcal{F}_i$, $ \forall i$. Therefore $A^c \in \cap_{i \in I} \mathcal{F}_i$
\item Let $A_1,A_2,\cdots \in \cap_{i \in I} \mathcal{F}_i$. Then $A_n \in \mathcal{F}_i$, $\forall i$, $\forall n$. Since $\mathcal{F}_is$ are $sigma$ algebras, we have $\cup_n A_n \in \mathcal{F}_i $, $\forall i$. Hence we have $\cup_n A_n \in  \cap_{i \in I} \mathcal{F}_i$
\end{enumerate}
This proof is the same even if the intersection is over an arbitrary index set because of the definition of intersection.
\\\\
To show that given $\Omega$ and a collection of subsets
$\mathcal{A}$, there exists a smallest $\sigma$ algebra.\\\\
Let $\mathbb{F}$ be the collection of all sigma algebras $\mathcal{F}_i$, $i\in I$ containing $\mathcal{A}$. Define $\sigma(\mathcal{A})=\cap_{i \in I} \mathcal{F}_i$. We have already shown that this is a $\sigma$ algebra. Clearly $\sigma(\mathcal{A}) \subset \mathcal{F}_i$, $\forall i$. Suppose there exists some $\mathcal{G} \subset \sigma(\mathcal{A})$ and which contains $\mathcal{A}$, then surely $\mathcal{G}\in \mathbb{F}$ and therefore $\mathcal{G}=\mathcal{F}_i$ for some $i$ which implies $\sigma(\mathcal{A}) \subset \mathcal{G}$. Hence we have $\sigma(\mathcal{A})=\cap_{i \in I} \mathcal{F}_i$ to be the smallest $\sigma$ algebra generated by $\mathcal{A}$

\end{proof}


\item[1.1.2 ]
\begin{proof}
To show $(\Omega, \mathcal{F}, \mathcal{P})$ is a probability space, we need to show $\mathcal{F} = \lbrace{A: A \text{ is countable or } A^c \text{ is countable}\rbrace}$ is a sigma algebra and $\mathcal{P}$ is a probability measure.\\\\
To show $\mathcal{F}$ is a sigma algebra, observe that:
\begin{enumerate}
\item
$\Omega \in \mathcal{F}$, since $\emptyset$ is countable by definition.
\item
if $A\in \mathcal{F}$ then $A$ must be either countable, in which case $(A^c)^c = A$ is countable and therefore $A^c$ belongs to $\mathcal{F}$, or $A^c$ is countable and so $A^c$ is also in $\mathcal{F}$.
\item
if $A_n \in \mathcal{F}, ~~\forall n \in \mathcal{N}$, then $A \triangleq \cup_{n=1}^\infty A_n$ is either countable, in which case it belongs to $\mathcal{F}$, or uncountable. If $A$ is uncountable then there exists an $m\in \mathcal{N}$ such that $A_m$ is uncountable and $A_m^c$ is countable. Since,  $A^c = (\cap_{n=1,n\neq m}^\infty A_n^c) \cap A_m^c$ and intersection of any set with a countable set is countable, $A^c$ is countable and thus $A$ belongs to $\mathcal{F}$.
\end{enumerate}
Thus $\mathcal{F}$ is a sigma algebra.\\\\
To show $\mathcal{P}$ is a probability algebra, observe that:
\begin{enumerate}
\item
Since $\Omega$ is uncountable $P(\Omega) = 1$.
\item
if $A_i \in \mathcal{F}$ are disjoint sets, then it is easy to see that either all the sets are countable or exactly one set is uncountable (because, if there are two disjoint sets $A_n$ and $A_m$ in $\mathcal{F}$ which are uncountable, then since $A_m \subseteq A_n^c$ and $A_n^c$ is countable, $A_m$ should be countable leading to contradiction). Thus, $P(\cup_{i=1}^\infty A_i) = \sum_{i=0}^{\infty} P(A_i) = 0$ if all $A_i$'s are countable and is equal to 1 if one of them is uncountable.
\end{enumerate}
Thus $\mathcal{P}$ is a Probability measure. 
\end{proof}

\item[1.1.3 ]
\begin{proof}
We shall show $\sigma(\mathcal{S}_d) = \mathcal{R}^d$ in multiple steps. 
\begin{enumerate}
\item
observe that,  $$(a_,b_1)\times(a_2,b_2)...\times(a_d,b_d) = \cup_{n\geq1}(a_,b_1-1/n]\times(a_2,b_2-1/n]...\times(a_d,b_d-1/n],$$ which implies that open rectangles $(a_,b_1)\times(a_2,b_2)...\times(a_d,b_d) \in \sigma(\mathcal{S}_d).$
\item \begin{claim}
Every open set in $\R^d$ is a countable union of open rectangles
\end{claim}
\begin{proof}
Associate with every internal point $x$ in the open set, an open rectangle with rational end points such that the open rectangle is a proper subset of the open set. This is possible as the rational points are dense in $\R$. Thus the claim is true as there are only countable such open rectangles and their union is equal to the open set.  
\end{proof}
\item Now, since every open set is a countable union of open rectangles, implying $\mathcal{R}^d \subseteq \sigma(\mathcal{S}_d)$.
\item
Observe that, $$(a_,b_1]\times(a_2,b_2]...\times(a_d,b_d] = \cap_{n\geq1}(a_,b_1+1/n)\times(a_2,b_2+1/n)...\times(a_d,b_d+1/n).$$
Since any open rectangle is also an open set and hence in $\mathcal{R}^d$, the above observation shows that a set of the form $(a_,b_1]\times(a_2,b_2]...\times(a_d,b_d] \in \mathcal{R}^d$, which together proves   
$\sigma(\mathcal{S}_d) \subseteq \mathcal{R}^d$

\end{enumerate}
\end{proof}

\end{enumerate}
\end{document}