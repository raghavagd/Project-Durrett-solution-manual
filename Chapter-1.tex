\documentclass[a4paper,english,10pt]{article}
\input{header}
\usepackage{etex,enumitem,hyperref,tikz,pgfplots}
%opening

\title{Solutions to Durrett's Probability: Theory and Examples \\(Edition 4.1)}
%\author{Deadline}
\date{}
\begin{document}
\maketitle
\section*{Chapter 1}
\begin{enumerate}
\item[1.1.2 ]
\begin{proof}
To show $(\Omega, \mathcal{F}, \mathcal{P})$ is a probability space, we need to show $\mathcal{F} = \lbrace{A: A \text{ is countable or } A^c \text{ is countable}\rbrace}$ is a sigma algebra and $\mathcal{P}$ is a probability measure.\\\\
To show $\mathcal{F}$ is a sigma algebra, observe that:
\begin{enumerate}
\item
$\Omega \in \mathcal{F}$, since $\emptyset$ is countable by definition.
\item
if $A\in \mathcal{F}$ then $A$ must be either countable, in which case $(A^c)^c = A$ is countable and therefore $A^c$ belongs to $\mathcal{F}$, or $A^c$ is countable and so $A^c$ is also in $\mathcal{F}$.
\item
if $A_n \in \mathcal{F}, ~~\forall n \in \mathcal{N}$, then $A \triangleq \cup_{n=1}^\infty A_n$ is either countable, in which case it belongs to $\mathcal{F}$, or uncountable. If $A$ is uncountable then there exists an $m\in \mathcal{N}$ such that $A_m$ is uncountable and $A_m^c$ is countable. Since,  $A^c = (\cap_{n=1,n\neq m}^\infty A_n^c) \cap A_m^c$ and intersection of any set with a countable set is countable, $A^c$ is countable and thus $A$ belongs to $\mathcal{F}$.
\end{enumerate}
Thus $\mathcal{F}$ is a sigma algebra.\\\\
To show $\mathcal{P}$ is a probability algebra, observe that:
\begin{enumerate}
\item
Since $\Omega$ is uncountable $P(\Omega) = 1$.
\item
if $A_i \in \mathcal{F}$ are disjoint sets, then it is easy to see that either all the sets are countable or exactly one set is uncountable (because, if there are two disjoint sets $A_n$ and $A_m$ in $\mathcal{F}$ which are uncountable, then since $A_m \subseteq A_n^c$ and $A_n^c$ is countable, $A_m$ should be countable leading to contradiction). Thus, $P(\cup_{i=1}^\infty A_i) = \sum_{i=0}^{\infty} P(A_i) = 0$ if all $A_i$'s are countable and is equal to 1 if one of them is uncountable.
\end{enumerate}
Thus $\mathcal{P}$ is a Probability measure. 
\end{proof}
\item[1.1.3 ]
\begin{proof}
We shall show $\sigma(\mathcal{S}_d) = \mathcal{R}^d$ in multiple steps. 
\begin{enumerate}
\item
observe that,  $$(a_,b_1)\times(a_2,b_2)...\times(a_d,b_d) = \cup_{n\geq1}(a_,b_1-1/n]\times(a_2,b_2-1/n]...\times(a_d,b_d-1/n],$$ which implies that open rectangles $(a_,b_1)\times(a_2,b_2)...\times(a_d,b_d) \in \sigma(\mathcal{S}_d).$
\item \begin{claim}
Every open set in $\R^d$ is a countable union of open rectangles
\end{claim}
\begin{proof}
Associate with every internal point $x$ in the open set, an open rectangle with rational end points such that the open rectangle is a proper subset of the open set. This is possible as the rational points are dense in $\R$. Thus the claim is true as there are only countable such open rectangles and their union is equal to the open set.  
\end{proof}
\item Now, since every open set is a countable union of open rectangles, implying $\mathcal{R}^d \subseteq \sigma(\mathcal{S}_d)$.
\item
Observe that, $$(a_,b_1]\times(a_2,b_2]...\times(a_d,b_d] = \cap_{n\geq1}(a_,b_1+1/n)\times(a_2,b_2+1/n)...\times(a_d,b_d+1/n).$$
Since any open rectangle is also an open set and hence in $\mathcal{R}^d$, the above observation shows that a set of the form $(a_,b_1]\times(a_2,b_2]...\times(a_d,b_d] \in \mathcal{R}^d$, which together proves   
$\sigma(\mathcal{S}_d) \subseteq \mathcal{R}^d$

\end{enumerate}
\end{proof}

\end{enumerate}
\end{document}